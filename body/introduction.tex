\section{Introduction}
% explain task we tackle
Multi-lingual word alignment (word alignment) is important to trasfer knowledge between the languages.
In contrast the number of language,
	only few language have enough resources to analyse and utlize them.
Solving multi-lingual word alignment task can be helpful to overcome the problem.
We focus on the unsupervised setting due to the motivation.

% explain difficulties in our task
The difficulties of word alignment arise from differences between languages.
For example,
	"Tonkatu" in japanese and english has different roles.
These differences are problematic noisy to estimate alignment between words in different languages.
\todo{explain more details}

% explain the difficulty/problem we focus on
One characteristic difference is frequency of words,
	which leads to different word embeddings.
In unsupervised settings, 
	because the word embeddings are the only resource we can use, 
	this problem tends to be a significant factor influencing the accuracy of alignment. 

% explain why the problem is important
All of these methods only consider the deficiency 
	in monolingual but pay no attention to the noise in bilingual alignment. 
There are two main difficulties in bilingual alignment:
	the first originated from different semantic of identical entity by cultural difference. 
Secondly, 
	high frequency words not always have exact correspondence with / between 
	the other language such as auxiliary word and symbol which have less information for alignment. 


% explain how previous work tackle the problem
To settle this problem, 
	(Conneau et, 2017) only use mutual nearest words pair 
	through training to generate orthogonal mapping. 
On the other side, 
	(David et, 2018) explicitly sets weight to decay rate 
	by frequency as hyper parameters to each language 
	to abate the influence of low frequency words. 


% explain our approach
In this research, 
	to settle this problem, 
	we use Sinkhorn algorithm (Sinkhorn, 1964) 
	to accustom weights of all words which can reduce transformation loss between two embedding sets. 
Our method achieves 
	remarkable result and can observably improve the robustness of model to different experiment settings.
\cmnt{I can't understand how this algorithm settle the noisy problem.}


% sum up our paper
The contributions to this paper are as follows :
\begin{itemize}
	\item We give a systematic introduction to Multilingual distribution models. 
	\item We present a novel model by modifying the weight to loss of each word automatically and achieve better result than previous State-of-art works.  
\end{itemize}





\begin{comment}

Multi-lingual word alignment is important to transferring knowledge between the languages 
	by the carrier of word, which is an application of domain adaption. 
\cmnt{
	[1] What is main problem in this paper? Domain adaption? Word align? or else? Unclear.
	[2] I can't understand what "by the carrier of word, which is an application of domain adaption" means.
	}
\todo{clarify main massage}
Based on the fundamental concept in cognitive language 
	that the physiological basis of human beings 
	plays an important role in formation of concepts and language (Add reference here), 
	even the disparate origin of the languages has led to barrier of comprehension, 
	same cognitive mechanism determine the feasibility to transfer the knowledge without supervision. 
\cmnt{
	[1]
	I can't understant grammar of this sentence.
	[2]
	Do not use unpractical words such as "cognitive language" and "physiological basis of human beings".
	[3]
	I can't understand what this sentense means.}
\todo{fix grammar}
For example, 
	with the cognitive concept that the capital is a center of a entity in country category, 
	" Tomboy " in French should have a high similarity with " London " in English. 
\cmnt{
	[1]Not use "cognitive concept".
	[2]I can't understand what this example explains.
	[3] What is the main problem? what is the role of this example?
	}
\todo{clarify them.}


There are several attempts on applying word distribution model in multi-lingual alignment model, 
	which we call it multi-lingual distribution model. 
\cmnt{
	[1] What is "word distribution model"?
	[2] What is the Research Question in this paper?
		"word distribution model" is related to it?
		Why you explain several attempts about it?
}
\todo{You MUST explain our motivations BEFORE describe technical facters.}


Word embedding has obtained huge success in wide range NLP downstream tasks 
	like sentiment classification and reading comprehension e. T C (Add reference here) 
\cmnt{[1]This sentense does not information}
On the other side, 
	multi-lingual distribution model aims at solving multi-lingual tasks ranging 
	from as machine translation to cross-lingual search engine, 
	which try to get several monolingual embeddings and their correspondences. 
\cmnt{[1] What is "On the other side" for? [2] I can't understand grammar. }
Multi-lingual distribution model has been proved to be useful 
	in cross-lingual transfer learning (Mikolov et al, 2013). 
\cmnt{No-information}
\cmnt{What this paragraph aims to explain? Non-information paragraph is unnecessary}
\todo{remove this paragraph}


Mikolov et al. (2013a) found that linear relationship exists 
	on two words embedding spaces by an empirical research, 
	and show learning the linear mapping with a small starting dictionary can overmatch the neural network.
\cmnt{How Mikolov's work related to our work??? I can't understand.}
As a further work, 
	orthogonal transformations have been shown effective 
	because its ability to keep distance invariance after transformation (Xing et al, 2015). 
\cmnt{What is "orthogonal transformations"? I can't understand.}

Nevertheless, 
	in realistic conditions, 
	there still exist some language pairs which don not have too much parallel data for learning. 
\cmnt{[1]What "Nevertheless" for ? [2] What is "realistic"? previous work is not realistic? [3] "there still exist some language pairs which don not have too much parallel data for learning" is our research question? really?}
To figure out the problem, 
	semi-supervised and unsupervised methods are developed recently. 
\cmnt{Non-information}
They have obtained obvious result even better than supervised methods. 
\cmnt{Non-information and wrong writing i.e., using vague adjective such as "obvious" and "better" to refer previous work}
We focus on full-unsupervised method of this research. 


However, 
	all of these methods only consider the deficiency 
	in monolingual but pay no attention to the noise in bilingual alignment. 
\cmnt{Too late explanation about noise. Noisy is core factor in our work}
There are two main difficulties in bilingual alignment, 
	the first originated from different semantic of identical entity by cultural difference. 
\cmnt{I recomment you to explain "There are two main difficulties" in second paragraph in introduction.}
For example, 
	different countries may have distinctive attitudes to an entity like color and person. 
\cmnt{I can't understand this example.}
Secondly, 
	high frequency words not always have exact correspondence with / between 
	the other language such as auxiliary word and symbol which have less information for alignment. 
\cmnt{What "Secondly" for? I can't understand what this sentence means. }
\end{comment}


